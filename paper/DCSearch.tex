% !TeX root = OptCuts.tex

%\section{Joint Discrete-Continuous Search}
\section{Search Procedure}%Jointly Discrete and Continuous Search}%<--- sounds weird
\label{sec:DCSearch}

Our primal update is a search procedure towards minimizing \eqref{eq:L} for a temporarily-fixed $\lambda$. 
% we will address details of how we update $\lambda$ below in Section\ \ref{}. 
Rather than trying to solve a potentially ill-conditioned real-valued optimization by approximating $T$ with $U$\ \cite{Poranne2017Autocuts}, we alternate local optimizations over $U$ and then $T$. %We augment a smooth descent search with a search over discrete topology. See Section~\ref{sec:DCSearch} for more details.%<--- these two sentences don't make sense, this is the topic of the current section

Our smooth descent steps apply standard continuous search. We locally minimize distortion $E_d$ over vertices $U$ while holding topology, $T$, fixed. Here we apply Newton-type method detailed in Section~\ref{sec:descentStep}.
%
Our topology descent step then applies a discrete search over local topology changes. We search neighboring topology to maximize a first-order reduction in the Lagrangian $L$ detailed in Section~\ref{sec:topologyStep}. 

In total, by ensuring monotonic decrease in $L$ over both the smooth and topology descent steps, we then prove that a near-stationary point can be reached for any input within a bounded number of alternations (Section~\ref{sec:convergence}). \danny{for varying $\lambda$ - i.e. our constrained problem seems like we need to spend a bit more work on this - see comments below.} 

\subsection{Motivation}

Our search procedure is essentially a graph search. However, due to the potentially huge computational cost, it is not possible to directly apply common graph search methods.

The UV topology space we are searching in can be represented using a directed graph $\mathcal{G}_T$ with its vertices $v_T \in \mathcal{V}_T$ being all possible UV topologies of a given 3D surface, and its edges $e_T \in \mathcal{E}_T$ are the local topological operations conducted on the UV map that can transform one UV topology to a nearby topology. Currently we support boundary and interior vertex splits, as well as corner vertex merge operations. \justin{add a small figure with these} 

We define the value $f_v$ of vertex $v^i_T$ as 
\[ f_v(v^i_T) = \min_{U} L(T^i, U) = E_s(T^i) + \min_{U} E_d(T^i, U) \]
and the weights $f_e$ of edge $e^{i,j}_{T}$ from $v^i_T$ to $v^j_T$ as 
\[ f_e(e^{i,j}_T) = f_v(v^j_T) - f_v(v^i_T). \]
Thus our problem could be restated as starting from an initial UV topology $v^0_T$ on the topology graph $\mathcal{G}_T$, search for a $v^i_T$ where $f_e(e^{i,j}_T) \geq 0$ is true for all $e^{i,j}_T \in \mathcal{E}^i_T$ ($\mathcal{E}^i_T$ is the set of all local topological operations in our dictionary that could be conducted on $v^i_T$).

Computing $f_v$ for one UV topology, however, requires a proper setting of initial UV coordinates plus a whole process of real-valued optimization. Moreover, even the number of neighbors of one UV topology is in the scale of $n_p$. These in practice prevent us from treating our problem as a usual graph search problem. Consequently, we construct a single search path on $\mathcal{G}_T$ and only approximate $f_v$ and $f_e$ in a topological first-order manner to guide the search along the path, where intermediate UV maps are also served as the initial points of the next search step.

\subsection{Primal Update}

In iteration $k$, given UV map $(T^{k-1}, U^{k-1})$ of the input surface mesh $M$, we conduct a joint discrete-continuous search to update both $T$ and $U$ towards minimizing $L$ (See Algorithm~\ref{alg:DCSearch}).

Similar to continuous search in smooth descent steps (Algorithm~\ref{alg:descentStep}), topology search is also composed of two stages: First decide a search direction (initiating a seam edit), and then compute a proper step size (extending the edit) to ensure sufficient energy decrease. There is no point in searching for every neighboring UV topology since, for example, initiating a cut in near-isometric regions will not help improve the map much. Thus we have a filtered candidate operation set $\hat{\mathcal{E}}^i_T$ for each topology descent step $i$ (Section~\ref{sec:topologyStep}). When deciding a direction, we fill the set with all promising operations produced by operation filtering (Section~\ref{sec:operationFiltering}), while when computing the step size, we fill then set with only the same operation as in the previous topology descent step for extending the step size.

\begin{algorithm}[h]
\SetAlgoLined
\KwData{$M$, $T^{k-1}$, $U^{k-1}$}
\KwResult{$T^k$, $U^k$}
$i \leftarrow 1$, $T^{k,1} \leftarrow T^{k-1}$, $U^{k,1} \leftarrow U^{k-1}$\;
$\hat{\mathcal{E}}^{k,1}_T \leftarrow$ operationFiltering(), $\delta^{k,1} = 0$\;
\Do{smooth descent step not converged}
{
	($T^{k,i+1}$, $U_a^{k,i}$) $\leftarrow$ topologyDescentStep($M$, $T^{k,i}$, $U^{k,i}$, $\hat{\mathcal{E}}^{k,i}_T$, $\delta^{k,i}$)\;
	($U^{k,i+1}$, $\delta^{k,i+1}$) $\leftarrow$ smoothDescentStep($M$, $T^{k,i+1}$, $U_a^{k,i}$)\;
	$\hat{\mathcal{E}}^{k,i+1}_T \leftarrow$ operations for extension\;
	$i \leftarrow i+1$\;
}
$T^k \leftarrow T^{k,i}$, $U^k \leftarrow U^{k,i}$
\caption{Primal Update $k$}
\label{alg:DCSearch}
\end{algorithm}

\subsection{Energy Definition}

To start with, we use symmetric Dirichlet energy~\cite{Smith2015Bijective} as our distortion energy, normalized by surface area:
\[ E_d = E_{SD} = \frac{1}{\sum_{t\in\mathcal{F}} |A_t|} \sum_{t\in\mathcal{F}} |A_t|(\sigma_{t,1}^2 + \sigma_{t,2}^2 + \sigma_{t,1}^{-2} + \sigma_{t,2}^{-2}) \]
where $\mathcal{F}$ is the set of all triangles, $|A_t|$ is the area of triangle $t$ on the input surface, and $\sigma_{t,i}$ is the $i$-th singular value of the deformation gradient of triangle $t$.

For seam energy, we simply use a normalized seam length:
\[ E_s = E_{SL} = \frac{1}{\sqrt{(\sum_{t\in\mathcal{F}} |A_t|)/\pi}} \sum_{i \in \mathcal{S}} |e_i| \]
where $\mathcal{S}$ is the set of all seam edges on the input surface, $|e_i|$ is the original length of edge $i$.

With the energies normalized, our $L$ is invariant of coordinate scale and resolution for meshes with the same shape.

\minchen{[TODO] fix iteration super script in Section~\ref{sec:descentStep} and \ref{sec:topologyStep}}

\section{Descent Steps for Continuous Optimization}

\subsection{Newton-type Iterations}

\begin{algorithm}[h]
\SetAlgoLined
\KwData{Input model, UV coordinates $U$, UV topology $v_T$}
\KwResult{$\arg\!\min_U E_{SD}$}
\For{each descent step inner iteration $j$}{
	compute $E_{SD}$ Hessian proxy $P^j$ using projected Newton\;
	compute $E_{SD}$ gradient $g^j$\;
	solve for search direction $p^j$ ($P^j p^j = -g^j$) using PARDISO symmetric indefinite solver\;
	compute initial step size $\alpha^j_0$ by avoiding element inversion\;
	backtracking line search with Armijo rule\;
	update $U^{j+1} = U^j + \alpha^j p^j$\;
	% read current\;
	% \eIf{understand}{
	% go to next section\;
	% current section becomes this one\;
	% }{
	% go back to the beginning of current section\;
	% }
}
\caption{Descent Steps}
\end{algorithm}

\subsection{Potential Accelerations for Practical Use}

Since our topological operations only change the mesh locally both on connectivity and coordinates, we could also update the Hessian or the decomposition locally after topology changes to save time. Besides, it's also interesting to try other Hessian approximation methods like L-BFGS or Majorization to explore further acceleration by finding a balance between computational cost and convergence rate.

For convergence tolerance of descent steps, $||\nabla E_{SD}||^2 \leq 10^{-6}$ (note that our energy is normalized) works generally well for all input models judging from the initiated fracture in the following topology step. In fact more inexact solve performs well on most of the models with even $||\nabla E_{SD}||^2 \leq 10^{-4}$, but some may result even better with $||\nabla E_{SD}||^2 \leq 10^{-8}$. Since we are conducting non-convex optimization, $||\nabla E_{SD}||^2$ is not always decreasing, which is also why we don't use Wolfe conditions for line search. The argument here for tolerance issue is that, it depends on whether we are truly in the infinitesimal region of a stationary. Some configuration with $||\nabla E_{SD}||^2 \leq 10^{-6}$ may still not inside the infinitesimal region of a stationary, where if optimization goes on, the $||\nabla E_{SD}||^2$ will go up and then fall down again to a real stationary, which is understandable in non-convex optimization.

% !TeX root = DCSearch.tex

\subsection{Topology Steps for Discrete Optimization}
\label{sec:topologyStep}

\subsubsection{Evaluating Topological Operations via Optimization on Local Stencils}

\begin{algorithm}[h]
\SetAlgoLined
\KwData{Input model, UV coordinates $U$, UV topology $v_T$}
\KwResult{A filtered set of UV vertices}
\eIf{boundary split}
{
  compute divergence of local gradients for all $n_{v,b}^i$ boundary vertices\;
  pick $(n_{v,b}^i)^{0.8}$ vertices with largest divergence as candidates\;
}
{
  compute divergence of local gradients for all $n_{v,i}^i$ interior vertices that doesn't connect to boundary\;
  pick $(n_{v,i}^i)^{0.8}$ vertices with largest divergence as candidates\;
}
\caption{Candidate Filtering}
\end{algorithm}

\begin{algorithm}[h]
\SetAlgoLined
\KwData{Input model, UV coordinates $U$, UV topology $v_T$, candidate UV vertices}
\KwResult{new UV topology $v_T$ and UV coordinates $U$}
\For{each candidate UV vertex}{
  \eIf{on boundary}
  {
    \For{each interior incident edge}{
      split and compute $\Delta E_{SD,l}$ locally\;
      compute $\Delta E_{w,l} = (1 - \lambda_t) \Delta E_{SD,l} + \lambda_t \Delta E_{se}$\;
    }
  }
  {
    \For{each pair of incident edges}{
      split and compute $\Delta E_{SD,l}$ locally\;
      compute $\Delta E_{w,l} = 0.5((1 - \lambda_t) \Delta E_{SD,l} + \lambda_t \Delta E_{se})$\;
    }
  }
}
\If{!interiorSplit}
{
  \For{each fracture tail}
  {
    merge the 2 incident boundary edges with averaged position\;
    \If{element inversion is detected}
    {
      project the averaged position to feasible region\;
      \If{feasible region is empty}
      {
        continue\;
      }
    }
    compute $\Delta E_{SD,l}$ locally\;
    compute $\Delta E_{w,l} = (1-\lambda_t)\Delta E_{SD,l}+\lambda_t \Delta E_{se}$
  }
}
conduct the operation with largest $|\Delta E_{w,l}|$
\caption{Local Evaluation}
\end{algorithm}
For boundary vertex that connects to another boundary, we free both the 2 boundary vertices while evaluating the local energy decrease.

\minchen{[TODO]
Since we are using the local estimation to approximate true global energy decrease, it would be necessary to find a balance between accuracy and efficiency by weighing between size of local stencils and number of optimization iterations to run.
}

\minchen{[NOTE]
Seamster's selected high curvature vertices are set to be the leaf of the seam tree while necessary, while our interior splitting scheme seems to always allow fracture to be extended on both the 2 sides of a picked vertex, which seems suboptimal. However, instead of just trying to connect interior vertex to the boundary like geometry images, we still have the reason to preserve our current interior splits because we've seen results that holes in UV map will be needed as a better solution for resolving some interior distortion. What we need to do is to take Seamster's inspiration and improve our interior splitting scheme.
}

\subsubsection{Line Search in Topological Space}

Once a fracture is initiated, it almost always be extended further in the later topology steps, which justifies the robustness of our method. To speed up the process, instead of waiting for another descent step and query all the filtered candidates again, we propagate this newly initiated fracture further in between the first several inner iterations of the following descent step.

After the fracture has been initiated, we first go to descent step to run an inner iteration and record the energy decrease $\Delta E_w^j$ and energy $E_w^{j,0}$. Then we evaluate $\hat{f}_v$'s for splitting the tail vertex of the newly initiated fracture along its incident edges. If the one with largest $\hat{f}_v$ satisfies $\hat{f}_v - E_w^{j,0} \leq \Delta E_w^j$, we propagate the fracture along this edge, and run another inner iteration to do another propagation query. If there's no propagation that could benefit more than running the inner iteration, we stop querying propagation in the current descent step and run inner iterations till convergence:

\begin{algorithm}[h]
\SetAlgoLined
\KwData{Input model, UV coordinates $U$, UV topology $v_T$}
\KwResult{new UV topology $v_T$ and UV coordinates $U$}
\For{each interior incident edge of current fracture tail vertex $k$}{
  split and compute $\Delta E_{SD, l}$ locally\;
  compute $\Delta E_{w,l} = (1 - \lambda_t) \Delta E_{SD,l} + \lambda_t \Delta E_{se}$\;
}
\eIf{$\max (|\Delta E_{w,l}|) \geq |(1-\lambda_t)\Delta E_{SD}^j|$}
{
  propagate fracture by splitting the vertex\;
  update fracture tail record\;
}
{
  turn off fracture propagation for the rest of the current descent step\;
}
\caption{Fracture Propagation Line Search}
\end{algorithm}

\minchen{[TODO] Write merge propagation}

\minchen{[NOTE] Besides fracture propagation line search, increase the depth of querying fracture initiation did the same thing but with more computational cost. However, it could improve the results because it's more global! (Inspired by Seamster)}

\subsubsection{Multiple Fracture Initiation}
\minchen{[TODO]}

Multiple fracture initiation would be redundant on a single cone-like region since only one fracture would be enough to release the distortion. However, initiating one fracture for each cone-like region within a single topology step would certainly accelerate the whole process.

This could be done by partitioning the UV space according to distortion or filtering measurement and initiate a valid fracture (if any) in each region in every topology step. The partition criteria could be separating the domain so that on each subdomain, the function is convex (has exactly one stationary).

\paragraph{Corner Merge} Applied relaxation method for projecting average position to inversion free position to start from, however cases are still there where only move the merged vertex is not enough
method ref:
S. Agmon. The relaxation method for linear inequalities
I. Eremin. The relaxation method of solving systems of inequalities with convex functions on the left sides

\paragraph{Forward Topological Line Search} It is based on a kind of sufficient energy decrease criteria similar to Armijo rule in the continuous setting, where our epsilon term is the energy decrease of the previous PN iteration.
To understand alternating lambda updates between PN iterations: lambda update changes the energy manifold, so it couldn’t be too frequent or the manifold can’t really be explored enough, nor it could be too seldom or it would be trapped in a local region.


\subsection{Operation Filtering}
\label{sec:operationFiltering}
As the number of vertex that could be splitted is in the scale of $n_p$, and initiating a cut in near-isometric regions will not help improve the UV map much, we filter the vertices to be considered in a split operation by computing the standard deviation of the local individual energy gradients on a vertex and only considering the top $n_p^{0.6}$ vertices with large deviation. \minchen{[TODO] why not filtering by energy? how to filter it well for bijective parameterization?}\justin{will this heuristic mess up the convergence theory?}

\subsection{Convergence}
\label{sec:convergence}

As our method is defined to guarantee convergence to a local optimum, we now analyze the convergence rate. First, $L$ monotonically decreases in each step. Now we look at smooth descent step $i$ and $i+1$, from $L^i \geq L^{i+1}$ we have
\[ E^i_{SD} - E^{i+1}_{SD} \geq \frac{1}{\lambda} (E^{i+1}_{se} - E^i_{se}) \geq \frac{1}{\lambda\sqrt{(\sum_t |A_t|)/\pi}} |e|_{min} \]
if we now only consider splitting operations that keep increasing $E_{se}$. $E_{SD}$ is lower-bounded theoretically by $4$, so we have
\[ n_{alter} \leq \frac{\lambda\sqrt{(\sum_t |A_t|)/\pi}}{|e|_{min}} (E^0_{SD} - 4) \]
The most important hint we can read from this is, to accelerate convergence, we can move through multiple vertices on $G_T$ in each topology descent step to increase $E^{i+1}_{se} - E^i_{se}$.

\minchen{[NOTE] seems not intuitive to generalize to involve merge operations}
\justin{nothing about this discussion seems to involve convergence rate.} \minchen{maybe change "rate" to "speed"?}