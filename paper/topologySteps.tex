% !TeX root = OptCuts.tex

\subsection{Topology Descent Steps}
\label{sec:topologyStep}

While each smooth descent step applies a complete Newton-type iteration with backtracking line search, we interleave topology descent steps with smooth descent steps to perform a forward line search in topology space. \justin{previous sentence sounds nice but I don't know what it means.} Topology search directions are first constructed and then their step size is \emph{increased}, rather than backtracked, to ensure energy decrease. Here step size refers to the search depth on $\mathcal{G}_T$, search direction means the selected topology at depth $1$. Given a candidate topology change, we evaluate its potential first-order reduction in $L$ by minimizing distortion, $E_d$, restricted to the local stencil of the edited seam edge and summed with the corresponding change in seam quality, $E_s$. Our experiments demonstrate that this local analysis is efficient and effective to explore seam edits (Section~\ref{sec:results_exp}).

In topology descent steps, starting from the UV map given by last smooth descent step, we move to a neighboring UV topology with the best first-order reduction in $L$, conducting a basic topological operation searched from the given operation set $\hat{\mathcal{E}}^i_T$. The UV coordinates of the edited seam vertices will also be updated. If the first-order reduction does not satisfy the threshold $\delta^i$ given by the last smooth descent step, however, nothing will be changed.

\begin{algorithm}[h]
\SetAlgoLined
\KwData{$M$, $T^i$, $U^i$, $\hat{\mathcal{E}}^i_T$, $\delta^i$}
\KwResult{$T^{i+1}$, $U_a^{i}$}
\For{each $e^{i,j}_{T}$ in $\hat{\mathcal{E}}^i_T$}{
  $\hat{f}_e(e^{i,j}_{T}) \leftarrow \Big(E^j_s + \lambda \min_{U^{i,j}} E_d(T^j,U)\Big) - E^i_{w}$\;
}
$U_a^{i} \leftarrow U^i$\;
\If{$\min_{e^{i,j}_{T} \in \hat{\mathcal{E}}^i_T} \hat{f}_e(e^{i,j}_{T}) \leq \delta^i$}
{
  $T^{i+1} \leftarrow T^j$, $U_a^{i,j} \leftarrow \argmin_{U^{i,j}} E_d(T^j,U)$\;
}
\caption{Topology Descent Step $j$}
\end{algorithm}
 
We compute the first-order reduction \justin{is this defined in an equation?}\minchen{it's in the pseudo-code} of $L$ from $T^i$ to $T^j$ for all $e^{i,j}_T \in \mathcal{E}^i_T$ in parallel, and only moves to \justin{didn't parse ``moves to''}\minchen{graph search..} $v^j_T$ with smallest $\hat{f}_e(e^{i,j}_{T})$, if this energy decrease is smaller than $\delta^i$. Recall that the threshold $\delta^i$ is set as the energy decrease in the current smooth descent step. For deciding a search direction, $\delta^i$ is near $0$ as the current smooth descent step converged, while for computing the topology descent step size, $\delta^i$ is not $0$. This together let our alternating process always pick the search step (continous or discrete) that decrease $L$ more to perform.

\paragraph{First-Order Reduction}
After making copies of the one-ring stencil of each candidate vertex to be split or merged, we split or merge them following all possible edge combinations in parallel, and optimize $E_d$ on them with the one-ring neighbors fixed. Since no more than 2 vertices are free to move, the system sizes only ranges from 2 to 4 degrees of freedom, which can be computed more efficiently rather than evaluating the true $f_v$ for deciding where to go next on $\mathcal{G}_T$.

For splits, the initial position of the split vertex can just be the original position before splitting. For corner merges, however, the merged vertex must have a locally-injective initial position\justin{neighborhood?}\minchen{what do you mean neighborhood?}. Instead of potentially minimizing a distance measure subject to a set of linear inequality constraints preventing element inversion, we first pose the merged vertex to the averaged position of the two vertices been merged. If element inversion is detected, we then apply relaxation method~\cite{Agmon1954Relaxation} to iteratively project from the averaged position to a inversion free position. However, cases are still there when only moving the merged vertex is not enough to obtain an inversion free initial local stencil. In these situations, we will just abandon the candidate, which rarely happens in practice and does not affect our result.\justin{didn't follow the last two sentences, maybe needs a small figure}

Splitting a boundary vertex along an edge that connects to another boundary vertex will remove a hole or produce an extra chart in our UV map. For this case, we free both the 2 boundary vertices while computing the energy decrease on the first-order stencil. \minchen{[TODO] 4-vertex merge to support joining two charts together, triangle moving operations?}