\section{Topology Steps for Discrete Optimization}

\subsection{Evaluating Topological Operations via Optimization on Local Stencils}

\begin{algorithm}[h]
\SetAlgoLined
\KwData{Input model, UV coordinates $U$, UV topology $v_T$}
\KwResult{A filtered set of UV vertices}
\For{each UV vertex}{
  compute divergence of local gradients\;
  independently picking $\sqrt{n_{v,b}^i}$ boundary vertices and $\sqrt{n_{v,i}^i}$ interior vertices with largest divergence as candidates\;
}
\caption{Candidate Filtering}
\end{algorithm}

\begin{algorithm}[h]
\SetAlgoLined
\KwData{Input model, UV coordinates $U$, UV topology $v_T$, candidate UV vertices}
\KwResult{new UV topology $v_T$ and UV coordinates $U$}
\For{each candidate UV vertex}{
  \eIf{on boundary}
  {
    \For{each interior incident edge}{
      split and compute $\Delta E_{SD,l}$ locally\;
      compute $\Delta E_{w,l} = (1 - \lambda_t) \Delta E_{SD,l} + \lambda_t \Delta E_{se}$\;
    }
  }
  {
    \For{each pair of incident edges}{
      split and compute $\Delta E_{SD,l}$ locally\;
      compute $\Delta E_{w,l} = 0.5((1 - \lambda_t) \Delta E_{SD,l} + \lambda_t \Delta E_{se})$\;
    }
  }
}
split the vertex with largest $|\Delta E_{w,l}|$
\caption{Local Evaluation}
\end{algorithm}

\textcolor{red}{
Do we need to treat compressed region and stretched region differently?
}

\textcolor{red}{
Since we are using the local estimation to approximate true global energy decrease, it would be necessary to find a balance between accuracy and efficiency by weighing between size of local stencils and number of optimization iterations to run.
}

\textcolor{red}{
Seamster's selected high curvature vertices are set to be the leaf of the seam tree, while our interior splitting scheme will allow fracture to be extended on both the 2 sides of a picked vertex, which seems suboptimal. However, instead of just trying to connect interior vertex to the boundary like geometry images, we still have the reason to preserve our interior splits because we've seen results that holes in UV map will be needed as a better solution for resolving some interior distortion. What we need to do is to take Seamster's inspiration and improve our interior splitting scheme.
}

\textcolor{red}{
Enable merge operation.
}

\subsection{Line Search in Topological Space}

Once a fracture is initiated, it almost always be extended further in the later topology steps, which justifies the robustness of our method. To speed up the process, instead of waiting for another descent step and query all the filtered candidates again, we propagate this newly initiated fracture further in between the first several inner iterations of the following descent step.

After the fracture has been initiated, we first go to descent step to run an inner iteration and record the energy decrease $\Delta E_w^j$ and energy $E_w^{j,0}$. Then we evaluate $\hat{f}_v$'s for splitting the tail vertex of the newly initiated fracture along its incident edges. If the one with largest $\hat{f}_v$ satisfies $\hat{f}_v - E_w^{j,0} \leq \Delta E_w^j$, we propagate the fracture along this edge, and run another inner iteration to do another propagation query. If there's no propagation that could benefit more than running the inner iteration, we stop query propagation in the current descent step and run inner iterations till convergence:

\begin{algorithm}[h]
\SetAlgoLined
\KwData{Input model, UV coordinates $U$, UV topology $v_T$}
\KwResult{new UV topology $v_T$ and UV coordinates $U$}
\For{each fracture tail vertex $k$}{
  \For{each interior incident edge of $k$}{
    split and compute $\Delta E_{SD, l}$ locally\;
    compute $\Delta E_{w,l} = (1 - \lambda_t) \Delta E_{SD,l} + \lambda_t \Delta E_{se}$\;
  }
  \eIf{the largest $|\Delta E_{w,l}|$ is larger than $|(1-\lambda_t)\Delta E_{SD}^j|$}
  {
    propagate fracture by splitting the vertex\;
  }
  {
    turn off fracture propagation for the rest of the current descent step\;
  }
}
\caption{Fracture Propagation Line Search}
\end{algorithm}

\textcolor{red}{Improve current propagation scheme by developing analogous line search method}

\textcolor{red}{Besides fracture propagation line search, increase the depth of querying fracture initiation did the same thing but with more computational cost. However, it could improve the results because it's more global!}

\textcolor{red}{
\subsection{Multiple Fracture Initiation}
}
Multiple fracture initiation would be redundant on a single cone-like region since only one fracture would be enough to release the distortion. However, initiating one fracture for each cone-like region within a single topology step would certainly accelerate the whole process. 

This could be done by partitioning the UV space according to distortion or filtering measurement and initiate a valid fracture (if any) in each region in every topology step. The partition criteria could be separating the domain so that on each subdomain, the function is convex (has exactly one stationary).
