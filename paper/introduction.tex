\section{Introduction}

\minchen{our method is in demand in CG community: }
Mesh parameterization is an important problem in computer graphics research because it has many applications, e.g. texture mapping, remeshing, detail transfer, etc. It has been investigated a lot in the past two decades \cite{some mesh parameterization papers}. The quality of a UV map is usually measured as it's isometric property, i.e. area and angle preservation without element inversions. Thus, discontinuities (seams) need to be introduced to help obtain maps with acceptable isometry. Previous works has been intensively focusing on the two steps separately, that is, where to place seams \cite{some seam placement papers}, and how to optimize for isometry (minimize distortion) with the seams provided \cite{some distortion minimization papers}. As the seam placement techniques are based on heuristics deduced from observation of the input shapes, they are not robust to all inputs and usually leads to suboptimal results.

We propose {\em OptCuts}, a joint discrete-continuous optimization framework that progressively edit seams in between distortion minimizations to automatically search for optimal UV map. We use a linear combination of symmetric Dirichlet energy~\cite{Smith2015Bijective} and normalized seam length as objective, of which the stationary w.r.t. both UV topology and coordinates are guaranteed to be reached within a bounded number of alternating iterations per balancing factor, input model, and initial embedding.
\minchen{[NOTE] (Here stationary w.r.t. UV topology is only in the approximation sense, because there might still be basic topological operations that could decrease the objective but end up not chosen because it's local evaluated energy decrease is not the largest one.)}

% terminology clarification if any
% ...

\minchen{main challenges: }
Seams, due to its discontinuous property, is not intuitive to be considered into traditional distortion minimization frameworks. What's more, in order for seams to be efficient, it needs to be sparse, which is another challenge for optimizing it with L2-type of distortion energies. The recently published AutoCuts~\cite{Poranne2017Autocuts} uses triangle soup data structure and model seams as a discontinous energy based on distance between corresponding edges. They jointly optimizes for symmetric Dirichlet energy and a smooth energy approximating their seam energy. In order to obtain sparse seams, they apply homotopy optimization to increase the accuracy of the approximation progressively. We observed that initially placing seams on all of the edges introduces multiple times of redundant degree of freedoms since during their solving process, most of the triangles keep the relative position to their neighbors. Besides, AutoCuts requires a certain amount of user guidance, e.g. parameter tuning, cut suggestion, patch movement, in order to obtain good results.

Based on our observations, we take the full mesh as starting point, and progressively place seams to help minimize distortion. We achieve this by proposing an optimization framework that alternates between discrete seam placement (in topology steps, Section~\ref{sec:topologyStep}) and continuous distortion minimization (in descent steps, Section~\ref{sec:descentStep}), so that we treat seam placement in a classic combinatoric way and also allow traditional distortion minimization methods to be directly adopted. In this way, we achieve a robust way to jointly optimize seams and distortion in a fully automatic manner while still allow users to communicate preferences on regional seam placement through edge weight painting (Figure~\ref{fig:edge_weight_painting}).

\minchen{better than Geometry Image and Seamster: }
Our framework is different from traditional seam cutting algorithms such as Geometry Image~\cite{Gu2002Geometry} and Seamster~\cite{Sheffer2002Seamster}, of which the core idea is to locate points of maximal currently predicted distortion and to add paths toward them. They do not perform well if no such obvious points exist, e.g. once distortion is distributed near-evenly across many surface points. Our framework in contrast searches for minimal cut elongation or shrinking steps that reduce the joint objective, thus we expect it to be more efficient in such settings (Figure~\ref{cases where there are not many obvious extremal points}).

\minchen{framework overview: }
Specifically, in descent steps, we minimize symmetric Dirichlet energy using projected Newton~\cite{Teran2005Robust} given the current UV topology. In topology steps, we search for a nearby UV topology that locally decrease the objective the most by querying a filtered set of basic topological operations including vertex split, edge merge, and triangle movement. To be appropriately aggressive on searching in the topological space, \minchen{[TODO] we develop a discrete forward tracking line search method to allow propagation of operation and multiple fracture initiation.}

\minchen{self-weighting: }
Since in application scenarios, an upper bound for distortion is more intuitive than picking a balancing factor, we also provide a constrained optimization formulation that seeks stationary w.r.t. both primal (UV coordinates and topology) and dual (balancing factor) variables subject to user specified distortion upper bounds. \minchen{[NOTE] we don't consider bounds on seam length cause it's less intuitive}

\minchen{other benefits of our framework: }
Seams created by our method are optimal for the distortion energy used. For conformal energy, it creates different set of seams that benefit conformality more (Figure~\ref{results of our method with conformal distortion energy}). In addition, our framework has many potential extensions to support more important mesh parameterization features, e.g. global bijectivity, seamlessness, etc.

\minchen{experiment overview: }
We demonstrate our framework's capabilities by comparing to AutoCuts and some typical classic seam cutting methods, including geometry image and Seamster. We show that given the same initial UV map, we reach same distortion bound with shorter seam length, and \minchen{[TODO] given UV outputs by other methods, our method can improve the distortion and seam placement. We also test our method on large scale inputs to show its scalability, and we use inputs with same shape but different triangulation to show that our method is invariant to mesh structure.}

\minchen{contribution: }
Our overall contribution is a novel framework that jointly optimizes seam placement and distortion for mesh parameterization, which incorporated into a constrained optimization formulation could allow users to obtain output UV map with specified upper bound on distortion. Key to our method are the topology search and dual variable treatment that allows optimal cuts to be searched with bounded distortion. Moreover, our framework has the potential to handle bijectivity and seamlessness as well. It also could be applied in other discrete-continuous geometry processing problems.
