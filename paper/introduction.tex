\section{Introduction}

% context
Mesh parameterization is an important problem in computer graphics research because it has many applications, e.g. texture mapping, remeshing, detail transfer, etc. The problem has been investigated a lot in the past two decades \cite{some mesh parameterization papers}. The quality of a UV map is usually measured as it's isometric property, i.e. area and angle preservation, and whether there are element inversions \cite{Sander2001Texture,Sheffer2005ABFPP}. Thus, due to the curvature and topology of the surfaces, discontinuities (seams) need to be introduced to help obtain maps with acceptable isometry.
Previous works has been intensively focusing on two topics separately, that is, where to place seams \cite{some seam placement papers}, and how to optimize for isometry (minimize distortion) with the seams provided \cite{some distortion minimization papers}. Since the seam placement techniques are based on heuristics deduced from observation of the input shapes, they are not robust to all inputs and usually leads to suboptimal results.

% task and object
We propose {\em OptCuts}, a joint discrete-continuous optimization framework that progressively place seams in between distortion minimizations to automatically search for optimal UV map. We use a linear combination of symmetric Dirichlet energy~\cite{Smith2015Bijective} and normalized seam length as objective, of which the stationary w.r.t. both UV topology and coordinates are guaranteed to be reached within a bounded number of iterations per balancing factor, input model, and initial embedding.
\minchen{[NOTE] (Here stationary w.r.t. UV topology is only in the approximation sense, because there might still be basic topological operations that could decrease the objective but end up not chosen because it is filtered out or its locally evaluated energy decrease is not the largest one.)}

% challenge
Seams, due to its discontinuous property, is not intuitive to be considered in traditional distortion minimization frameworks. Moreover, in order for seams to be efficient, it needs to be sparse, which is another challenge for optimizing it with L2-type distortion energies. The recently published AutoCuts~\cite{Poranne2017Autocuts} model seam as a discontinuous energy using triangle soup data structure and jointly optimize it with distortion via homotopy optimization. We observed that initially placing seams on all the edges introduces multiple times of redundant degree of freedoms since during their solving process, most of the triangles keep the relative position to their neighbors. Besides, since the placement of seams highly depends on the homotopy path, AutoCuts requires a certain amount of user guidance, e.g. parameter tuning, cut suggestion, patch movement, in order to obtain good results.

Instead, we optimize seams in a combinatoric way and allow traditional distortion minimization to be directly adopted. As distortion is usually minimized by searching in the UV coordinate space, we construct another search path in the UV topology space: starting from an initial embedding, in each topology search iteration we first find a search direction that locally decrease the objective the most (in topology step, Section~\ref{sec:topologyStep}); and then we conduct a newly derived forward-tracking line search scheme alternated with distortion minimization iterations to decide the step-size (in descent step, Section~\ref{sec:descentStep}).

% compared to Geometry Image and Seamster
Our framework is different from traditional seam cutting algorithms such as Geometry Image~\cite{Gu2002Geometry} and Seamster~\cite{Sheffer2002Seamster}, of which the core idea is to locate points of maximal currently predicted distortion and to add paths toward them. They do not perform well if no such obvious points exist, e.g. once distortion is distributed near-evenly across many surface points. Our framework in contrast searches for minimal cut elongation or shrinking steps that reduce the joint objective, thus we expect it to be more efficient in such settings (Figure~\ref{cases where there are not many obvious extremal points}).

% self-weighting
Although we acquire adaptivity to various inputs by normalizing the energy, it is still not intuitive for users to directly communicate all expectations through a balancing factor in the objective. Consequently, we also provide a constrained optimization formulation that seeks stationary w.r.t. both primal (UV coordinates and topology) and dual (balancing factor) variables subject to user specified distortion upper bounds (Section~\ref{sec:self_weighting}).

% experiments
We demonstrate our framework's capabilities by comparing to AutoCuts and some typical classic seam cutting methods, including geometry image and Seamster (Section~\ref{sec:results}). Given the same initial UV map, we reach same distortion bound with shorter seam length efficiently. We also test our method on large scale inputs to show its scalability, and we use inputs with same shape but different triangulation to show that our method is invariant to mesh structure.
Although OptCuts doesn't need any user assistance, it still allows users to communicate preferences on regional seam placement through edge weight painting (Figure~\ref{fig:edge_weight_painting}).
In addition, our seams are optimal for the distortion energy used. For example, it creates different set of seams that benefit conformality more if conformal energy is used (Figure~\ref{results of our method with conformal distortion energy}).

% contribution
Our overall contribution is a novel framework that jointly optimizes seam placement and distortion for mesh parameterization, which incorporated into a constrained optimization formulation could allow users to obtain UV map with specified upper bound on distortion. Key to our method are the UV topology search and dual variable treatment that allows optimal cuts to be searched with bounded distortion. Moreover, our framework has the potential to handle global bijectivity and seamlessness as well. It also could be applied in other discrete-continuous geometry processing problems (Section~\ref{sec:conclusion}).
